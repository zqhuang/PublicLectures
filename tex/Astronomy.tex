\documentclass[CJK]{beamer}
\input{macros.tex}
\title{Department of Astronomy}
\author{Zhiqi Huang}
\institute{Sun Yat-sen University}
\date{July 11, 2017}


\begin{document}


\begin{frame}
  \bch
  \bcenter
      {\Huge 天文系简介}

      \addfig{0.6}{telescope.png}

      {\large 黄志琦}

      中山大学物理与天文学院 
      
      2017年7月11日

      
      \ecenter
      \ech
\end{frame}



\begin{frame}
\chtitle{内容摘要}
\centering
\bch
\bitem
\item{天文学尺度}
\item{太阳系和地外行星}
\item{恒星}
\item{致密天体}
\item{星系和宇宙学} 
  \eitem
  \ech
\end{frame}

\section{Astronomical Scales}


\tpage{天文学尺度}


\begin{frame}
  \chtitle{形象地丈量长度的办法:熟知的速度$\times$时间}
  \bch
  \bitem
\item{从巢酒店到叠石酒家,大概2分钟步行路程。}
\item{从珠海中大校区到扬名广场海底捞,大概20分钟车程。}
\item{从珠海到北京全聚德,大概4小时飞机。}
  \eitem

  
  \ech
\end{frame}

\begin{frame}
  \chtitle{从中国到美国}
  \bch
  飞机从中国飞到美国,大概需要十几小时。
  \addfig{1.5}{plane.jpg}
  而跑得最快的光,大概只需要几十毫秒。
  \addfig{1.5}{light.jpg}
  \ech
\end{frame}


\begin{frame}
  \chtitle{地月距离}
  \bch
  月球是地球的卫星
  
  \addfig{3}{earth_to_moon.jpg}
 \bcenter 
  从地球到月球,光要跑大约1秒。
  \ecenter
  \ech
\end{frame}


\begin{frame}
  \chtitle{太阳系的行星}
  \bch
  地球是太阳系的一颗行星。
  
  \addfig{3.6}{solar-system.jpg}
 \bcenter 
 从太阳到地球,光要跑大约8分钟,即使跑到最远的海王星也只要大概几个小时。
  \ecenter
  \ech
\end{frame}


\begin{frame}
  \chtitle{附近的其他“太阳系”}
  \bch
  \addfig{3.3}{sun_neighbors.png}
 \bcenter 
 光跑到离太阳最近的另一个”太阳“(恒星)大概需要4年。
  \ecenter
  \ech
\end{frame}


\begin{frame}
  \chtitle{银河系}
  \bch
  \bmini{0.7}
  \addfig{2.6}{Milkway.jpg}
  \emini
  \bmini{0.26}
  太阳系在银河系内;
  
  银河系里大概有一千亿个“太阳”(恒星)

  \skipline
  
  光从银河系一边跑到另一边要10万年以上。
  \emini

  {\scriptsize 如果把太阳系比做一枚硬币,银河系大概有中国版图那么大。}

  \ech
\end{frame}


\begin{frame}
  \chtitle{河外星系}
  \bch
  \bmini{0.7}
  \addfig{2.6}{galaxies.jpg}
  \emini
  \bmini{0.26}
  可观测宇宙中大概有一千亿个”银河系“(星系)

  \skipline
  
  我们看到的来自可观测宇宙的“边缘”的光大概已经跑了一百三十多亿年。
  \emini

  {\scriptsize 如果把银河系比做一块瓷砖,可观测宇宙大概有中国版图那么大。}
  
  \ech
\end{frame}

\tpage{天文学研究的对象小到行星,大到整个可观测宇宙,横跨约20个数量级!}

\section{Solar System and Exoplanets}

\tpage{太阳系与地外行星}


\begin{frame}
  \chtitle{观测到的地外行星与日俱增}
  \bch
  \addfig{3.8}{exoplanets_discovery.png}
  
  \ech
\end{frame}
  
\begin{frame}
  \chtitle{行星的形成和演化}
  \bch
  \bitem
\item{行星是如何形成的?}  
\item{地球很特殊吗?其他“太阳系”中有多少类地行星?地外智能生命存在的可能性有多大?}
  \eitem

  \skipline
  
  \bmini{0.45}
  \bcenter
  余聪教授

  \lfig{1}{yucong.jpg}

  \ecenter
  
  \emini
  \bmini{0.5}
  \bitem
\item{{\bf Planet Migration}}
\item{{\bf Formation of Super-Earth}}
\item{{\bf Tides in Hot Jupiter}}
  \eitem
  \emini
  \ech
\end{frame}

\section{Stellar Evolution}

\tpage{恒星}

\begin{frame}
  \chtitle{恒星的演化历史}
  \bch
  \bmini{0.7}
  \lfig{2.6}{HRDiagram.png}
  \emini
  \bmini{0.25}
  \bcenter
  分子云塌缩
  $$ \Downarrow$$
  主序星
  $$ \Downarrow$$
  红巨星
  $$ \Downarrow$$
  白矮星
  \ecenter
  \emini
  \ech
\end{frame}

\begin{frame}
  \chtitle{张泳教授}
  \bch
  \bmini{0.3}
  \lfig{1}{zhangyong.jpg}

        {\bf 恒星死亡后星周包层中的物理和化学过程}
        
        \lfig{1}{latestellar.png}
  
  \emini
  \bmini{0.66}
        \lfig{0.8}{molecule.png}         \lfig{1}{molecule2.png}
        \bitem
        \item{星风抛射过程、与星际介质的相互作用}
        \item{搜寻星周分子(特别是复杂分子)}
        \item{研究分子在星周演化中的化学过程及其对星际物质的增丰}
        \item{利用研究光致电离气体星云观测光谱研究极端环境下的等离子体物理}
        \eitem
  
  \emini
  \ech
\end{frame}

\begin{frame}
  \chtitle{姜晨研究员 (asteroseismology)}
  \bch
  \bmini{0.3}
  \lfig{1}{jiangchen.jpg}
  \emini
  \bmini{0.65}
  研究兴趣:
  
  Solar-like oscillations, Stellar evolution, Stellar modelling, Stellar inner structure, Photometric data analysis, Low-mass stars, Red giant stars, and Exoplanets
  \emini
  \ech
\end{frame}

\section{Dense Object}

\tpage{致密天体(白矮星,中子星,黑洞)}

\begin{frame}
  \chtitle{恒星塌缩的产物:白矮星,中子星和黑洞}
  \bch
  
  当恒星燃尽,没有足够的热压强抗衡引力,就会塌缩
  
  \bitem
\item{小质量恒星 $\Rightarrow$  白矮星 (电子简并压抗衡引力)

\addfig{0.8}{whitedwarf.jpg}
  }
\item{中等质量恒星 $\Rightarrow$ 中子星 (中子简并压抗衡引力)

\addfig{0.8}{neutronstar.jpg}  }
\item{大质量恒星 $\Rightarrow$ 黑洞 (塌缩为奇点)
    \addfig{0.8}{blackhole.jpg}}
  \eitem
  
  \ech
\end{frame}

\begin{frame}
  \chtitle{余聪教授}
  \bch
  \bmini{0.6}
        {\bf Magnetars}

        Magnetic Flux Tubes

        High energy outburst

        \skiplines
        
        {\bf Intermittent Pulsars}

        Twisted magnetosphere

        State Transition

        \skiplines
        
        {\bf Jets in Kerr Spacetime}

        Magnetically Dominated Poynting Jets
  
  
  \emini
  \bmini{0.35}
  \lfig{0.8}{magnetar.jpg}

  \skipline

  \lfig{0.8}{pulsar.jpg}

  \skipline

  \lfig{0.8}{kerrBH.jpg}  
  \emini
  \ech
\end{frame}

\begin{frame}
  \chtitle{申荣锋副教授}
  \bch
  \bmini{0.35}
  
  \addfig{1}{shenrongfeng.jpg}
  
  爆发性的天文暂现源:伽马射线暴,超大质量黑洞潮汐瓦解恒星事件,超高光度X射线源,超新星;黑洞吸积盘;致密星
  \emini
  \bmini{0.6}
  \bcenter
  Tidal Disruption Events

  \lfig{2}{TDE.png}
  \ecenter
  \emini
  \ech
\end{frame}

\begin{frame}
  \chtitle{其他致密天体相关}
  \bch
  \bitem
\item{{\bf 谭柏轩副教授}: 高能天体物理丶伽玛射线天文学丶中子星丶毫秒脉冲星丶X射线双星丶伽玛射线双星丶伽玛射线暴。}
\item{{\bf 张雪光研究员}: 活动星系核,大质量黑洞}
\item{{\bf 艾艳丽研究员}: 活动星系核, 类星体特性  }  
\item{{\bf 张福鹏研究员}: 超大质量黑洞,广义相对论,活动星系核}  
\item{{\bf 杨怡蓉研究员}: 中等质量黑洞、超亮X射线源、X射线双星/瞬变星、吸积毫秒脉冲星 }
\item{{\bf 王静研究员}: Accreting compact objects and its cosmological applications; Graviton in neutron star binary systems}
  \eitem
  \ech
\end{frame}


\section{Galaxy and Cosmology}

\tpage{星系和宇宙学}

\begin{frame}
  \chtitle{宇宙的大尺度结构}
  \bch
  从大尺度上看,宇宙大致上是均匀且各向同性的,但又存在微小的密度起伏。

  \addfig{3}{2MASS.jpg}


\bcenter
2MASS Extended Source Catalog

\skipline
  宇宙学的研究对象: {\bf 宇宙中的物质组成和来源,以及大尺度结构的起源和演化。}  
\ecenter
  \ech
\end{frame}

\begin{frame}
  \chtitle{宇宙学数值模拟小组({\bf 冯珑珑教授, 林伟鹏教授}, 朱维善, 张雪光, 艾艳丽,张福鹏,姜晨,汪洋)}
  \bch
  \bmini{0.5}
  DM + baryon simulations
  \emini
  \bmini{0.45}
  \lfig{1.5}{tianhe.png}
  \emini

  \addfig{3.2}{NBodySim.png}
  \ech
\end{frame}


\begin{frame}
  \chtitle{冯珑珑教授,林伟鹏教授等策划建6.5米望远镜}
  \bch
  \addfig{1}{telescope.png}
  
  \bmini{0.48}
  冯珑珑教授
  
  \addfig{1.}{fenglonglong.jpg}
  \emini
  \bmini{0.48}
  林伟鹏教授  
  \addfig{1.}{linweipeng.jpg}
  \emini

  \ech
\end{frame}


\begin{frame}
  \chtitle{李淼教授}
  \bch
  \bmini{0.3}
  \lfig{1.3}{limiao.jpg}

  宇宙学,弦论,高能物理
  \emini
  \bmini{0.65}
  \lfig{2.7}{holographicDE.png}
  \emini
  \ech
\end{frame}


\begin{frame}
  \chtitle{黄志琦教授:宇宙背景微波辐射}
  \bmini{0.2}
  \addfig{0.7}{huangzhiqi.jpg} 
  \emini
  \bmini{0.3}
  \bitem
\item{CMB}
\item{modified gravity}
\item{Inflation}
\item{preheating}
  \eitem
  \emini
  \bmini{0.45}  
  \lfig{2.}{plancksatellite.jpg}
  \emini

  \bmini{0.45}
  \lfig{2}{planck2014_TT_Dl_NORES_bin30_w88mm.pdf}
  \emini
  \bmini{0.5}
  \lfig{2}{ors_Q_on_Tmax_oriented.pdf}
  \emini
\end{frame}

\begin{frame}
  \chtitle{高显教授:Galileon/Horndeski theory and beyond}
  \bch
  \bmini{0.2}
  \lfig{0.8}{gaoxian.jpg}
  
  宇宙学、引力
  \emini
  \bmini{0.76}
  \lfig{3.5}{gaoxianwork.png}
  \emini
  \ech
\end{frame}

\begin{frame}
  \chtitle{其他宇宙学相关}
  \bch
  \bitem
\item{{\bf 林树教授}: AdS/CFT对偶(引力/规范对偶),夸克胶子等离子体,有限温度/密度场论,量子色动力学}
\item{{\bf 孙佳睿副教授}: 引力论和场论 }
\item{{\bf 王爽研究员}: 超新星系统误差;宇宙加速膨胀的数值研究}  
\item{{\bf 李明华研究员}: 星系、大尺度结构形成,星系团宇宙学}
  \eitem
  
  \ldots
  
  \ech
\end{frame}

\begin{frame}
  \chtitle{总结}
  \bch
  \addfig{3}{summary.png}
  \ech
\end{frame}

\begin{frame}
\chtitle{欢迎加入中山大学物理与天文学院}
\bch
\centering

\addfig{3}{schoolprofile.png}

{\large \bf http://spa.sysu.edu.cn}

\ech
\end{frame}

\end{document}



