\documentclass[CJK]{beamer}
\input{macros.tex}
\title{The Ultimate Distance - Modern Observational Cosmology}
\author{Zhiqi Huang}
\institute{Sun Yat-sen University}
\date{2017-09-08}


\begin{document}


\begin{frame}
  \bch
  \bcenter
      {\blue \Large 诗和远方 - 现代天文学观测之极致}

      {\blue \large The Ultimate Distance - Modern Observational Astronomy}      

     \lfig{0.6}{telescope.png}

      {\large 黄志琦 教授, 中山大学 物理与天文学院}
      
      { Prof. Zhiqi Huang, Sun Yat-sen University}      

      \skiplines
      
      全球创新者大会(GIC 2017)

      \skipline
      
      2017-09-08

      
      \ecenter
      \ech
\end{frame}



\begin{frame}
\chtitle{内容摘要(Outline)}
\centering
\bch
\bitem
\item{浩瀚的宇宙 (The Deep Universe)}
\item{野心勃勃的天文学家们 (The Ambitious Astronomers)}  
\item{第二个地球的候选者 (Potentially Habitable Exoplanets)}
  \eitem
  \ech
\end{frame}

\section{The Deep Universe}

\begin{frame}
  \chtitle{天文学尺度 Cosmological Scales}
  \bch
  \addfig{3}{galaxy_zoom.jpg}

  {\small
   $$\frac{\text{size of solar system}}{\text{size of Milky Way}} \approx \frac{\text{size of Milky Way}}{\text{size of observable Universe}} \approx \frac{\text{size of a coin}}{\text{size of China}}$$}
  \ech
\end{frame}

\begin{frame}
  \chtitle{宇宙的韵律:重子声波振荡 (Baryon Acoustic Oscillations)}
  \bch

  The ``rhythm'' of the galaxy distribution is from vacuum quantum fluctuations in early Universe.

  
  宇宙星系分布的起伏韵律:来自宇宙早期的真空量子波动。

  \addfig{2}{BAO.jpg}
  
  \ech
\end{frame}

\begin{frame}
  \chtitle{来自最远方的光(Lights from the Edge of Observable Universe)}
  \bch
\bcenter
  cosmic microwave background
  
  \lfig{3}{PlanckFullsky.png}

  The lights are from about 14000000000 years ago and 400000000000000000000000 kilometers away.
  \ecenter
  \ech
\end{frame}


\section{The Ambitious Astronomers}

\begin{frame}
  \chtitle{下一代天文观测(The next-generation observations)}
  \bch
  space telescope and ground-based telescopes (空间和地面望远镜)
  
  measuring $\sim$ billion galaxies (测量数以亿计的星系)
  \bmini{0.4}
  \lfig{1.5}{Euclid.jpg}
  \emini
  \bmini{0.55}
  \lfig{2.3}{LSST.jpg}
  \emini
  \ech
\end{frame}

\begin{frame}
  \chtitle{引力波天文学 (Gravitational Wave Astronomy)}  
  \lfig{1}{wd_gw.jpg}\lfig{3}{tianqin.jpg}
\end{frame}

\section{Potentially Habitable Exoplanets}

\begin{frame}
  \chtitle{系外行星的探测 (exoplanets catalog)}
  \addfig{4}{exoplanets_scatter.png}
\end{frame}


\begin{frame}
  \chtitle{第二个地球的候选者(Potentially Habitable Planets)}
  \addfig{4.2}{HEC_All_Distance.jpg}
\end{frame}
  

\begin{frame}
  \chtitle{欢迎访问中山大学物理与天文学院 (School of Physics and Astronomy, Sun Yat-sen Universe)}
\bch
\centering

\addfig{3}{schoolprofile.png}

{\large \bf http://spa.sysu.edu.cn}

\ech
\end{frame}

\end{document}



