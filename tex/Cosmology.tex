\documentclass[CJK]{beamer}
\input{macros.tex}
\title{Advanced Topics in Physics and Astronomy -- Cosmology in a Nutshell}
\author{}
\date{}


\begin{document}

\begin{frame}
 
\begin{center}
\bch
\begin{Large}
物理学与天文学前沿系列

\skipline
Cosmology in a Nutshell

\skipline
宇宙学精要 


\end{Large}
\skipline
黄志琦  

huangzhq25@sysu.edu.cn
\ech
\end{center}
\end{frame}


\section{Cosmology and Quantum Gravity}
\begin{frame}
\chtitle{宇宙学试图回答的第一个问题}
\bch
天下万物从何而来?
\ech
\end{frame}

\begin{frame}
\chtitle{最早的宇宙学家李聃说……}
\bch
\begin{minipage}{0.45\textwidth}
\includegraphics[width=1.6in]{laozi.jpeg}
\end{minipage}
\begin{minipage}{0.45\textwidth}
天下万物生于有,有生于无
\end{minipage}
\ech
\end{frame}

\begin{frame}
\chtitle{}
\bch
经过几千年的探索,我们发现:李聃老师蒙对了!
\ech
\end{frame}

\begin{frame}
\chtitle{真空不空}
\bch
\begin{minipage}{0.5\textwidth}
场无处不在
\skipline
每种场对应一种粒子(或一对正反粒子)。例如电磁场对应光子,Dirac场对应正负电子。
\end{minipage}
\begin{minipage}{0.4\textwidth}
\includegraphics[width=1.5in]{magnet.jpeg}
\end{minipage}
\ech
\end{frame}

\begin{frame}
\chtitle{物理真空的定义}
\bch
\begin{minipage}{0.35\textwidth}
\includegraphics[width=1.2in]{vacuum.jpeg}
\end{minipage}\begin{minipage}{0.55\textwidth}
当所有场都处于最低能量态(即0个粒子的态)的时候,就是物理上定义的真空。
\end{minipage}

\skipline
\begin{equation}
\begin{array}{ccclr}
- & - & - & \mathrm{3\ particles} &\\
- & - & - & \mathrm{2\ particles}&\\
- & - & - & \mathrm{1\ particle} &\\
- & - & - & \mathrm{0\ particles}& \rightarrow {\bf \rm vacuum}\\ 
{\rm field\ A} & {\rm field\ B} & {\rm field\ C}
\end{array} \nonumber
\end{equation}
{\scriptsize
注:本图的上面几行仅为示意,实际上一个粒子可以拥有比两个粒子更多的能量。
}

\ech
\end{frame}

\begin{frame}
\chtitle{“无中生有”没有你想象得那么容易}
\bch

无论是经典理学还是量子力学都要求能量守恒。从最低能态变到不是最低的能态,势必破坏能量守恒。

\skipline
所以正常情况下真空里什么都不会产生!

\ech
\end{frame}

\begin{frame}
\chtitle{这个道理很显然}
\bch

你在家葛优躺啥事不干,正常情况下不会发财。

\ech
\end{frame}


\begin{frame}
\bch
除非你家房子地段好,价值飙升。
\ech
\end{frame}

\begin{frame}
\chtitle{粒子场住的“房子”}
\bch
各种粒子场都“住”在四维时空里。如果时空结构发生剧烈变化,就能“无中生有”。

\skipline

基于这种想法的理论:
\begin{itemize}
\item{霍金的黑洞蒸发理论}
\item{宇宙早期暴涨(Inflation)理论}
\end{itemize}
\ech
\end{frame}

\begin{frame}
\chtitle{等一下,我还有问题}
\bch
\begin{itemize}
\item{早期宇宙时空为何会暴涨?}
\item{不是说量子场论和广义相对论还没统一吗?那怎么计算时空剧烈变化时的各种量子场的行为?}
\end{itemize}
\ech
\end{frame}

\begin{frame}
\chtitle{为何暴涨}
\bch
第一个问题100年前已经被爱因斯坦解决了:

\begin{minipage}{0.45\textwidth}
\includegraphics[width=1.2in]{einstein.jpeg}
\end{minipage}
\begin{minipage}{0.45\textwidth}
$$ M_p^2 G^{\mu\nu} =  T^{\mu\nu} - \rho_{\rm vacuum} g^{\mu\nu} $$
广义相对论里的Einstein方程:真空能和标量场的势能都具有排斥作用会使空间加速膨胀。

\end{minipage}
\ech
\end{frame}

\begin{frame}
\chtitle{怎么计算弯曲时空里的量子场}
\bch
关于第二个问题,有一个好消息和一个坏消息。
\ech
\end{frame}

\begin{frame}
\chtitle{好消息和坏消息}
\bch
\begin{minipage}{0.6\textwidth}
坏消息:我们无法从第一原理出发计算黑洞蒸发或者早期宇宙的时空暴涨。
\end{minipage}
\begin{minipage}{0.3\textwidth}
\includegraphics[width=0.8in]{baozou_haha.png}
\end{minipage}

\skipline
\skipline

好消息:霍金以及一批俄罗斯科学家发展起来了一种在特殊情况下可以从平直时空外推到弯曲时空的计算方法。
\ech
\end{frame}


\begin{frame}
\chtitle{有趣的是:这套计算方法管用}
\bch
现代宇宙学的大量观测验证了这套计算方法非常管用。

\includegraphics[width=1.5in]{planck_masked.pdf}\includegraphics[width=1.in]{sdss.jpeg}\includegraphics[width=1.5in]{lymanalpha.jpg}

现在的宇宙学标准模型里只有6个自由参数。我们迄今已经检查了数百亿个观测数据,没有发现任何和理论预言矛盾的地方!

\skipline
\skipline
那么,这套计算方法为什么管用呢?
\ech
\end{frame}


\begin{frame}
\chtitle{物理学革命呼之欲出?}
\bch
\begin{minipage}{0.6\textwidth}
谁知道它为什么管用!
\end{minipage}
\begin{minipage}{0.3\textwidth}
\includegraphics[width=0.8in]{baozou_haha.png}
\end{minipage}

\skipline
\skipline
\skipline
\skipline
还记得一百年前大家知道怎么算波尔半径,光电效应等等,却不知道为什么能这样算的情形么?
\ech
\end{frame}


\begin{frame}
\chtitle{讲到这里,大家觉得宇宙学有趣么?}
\bch
其实我的内心非常忐忑……
\ech
\end{frame}


\begin{frame}
\chtitle{院里领导非常重视本科生教学}
\bch
\skipline
\begin{minipage}{0.4\textwidth}
力学课是一切物理课的基础,如果学生将来对物理失去了兴趣,我要承担责任……

\vskip 2in
\ 
\end{minipage}
\includegraphics[width=1.8in]{miaoshu.jpg}
\ech
\end{frame}


\begin{frame}
\chtitle{我深入学习领会领导的讲话}
\bch
\skipline
\begin{minipage}{0.4\textwidth}
宇宙学不是基础学科,如果学生将来对物理失去了兴趣,跟我没关系……
\vskip 2in
\ 
\end{minipage}
\includegraphics[width=1.6in]{myself.jpg}
\ech
\end{frame}


\begin{frame}
\chtitle{即便如此,我还是很认真地备了课的……}
\bch
\includegraphics[width=1.9in]{conversation.png}
\ech
\end{frame}


\begin{frame}
\chtitle{我还曾特地找你们的林树班主任取过经}
\bch
\begin{minipage}{0.4\textwidth}
\includegraphics[width=1.5in]{linshu.jpg}
\end{minipage}
\begin{minipage}{0.55\textwidth}
\begin{scriptsize}
几周前,林老师刚给你们上完物理前沿课——
\skipline

我:感觉咋样?
\skipline

林老师:呃,他们好像都没怎么听懂。
\skipline

我:不会吧,本科生基础这么差?你讲了什么?
\skipline

林老师:夸克胶子等离子态。
\skipline

我:\includegraphics[width=0.5in]{wuyu.jpg}\includegraphics[width=0.5in]{wuyu.jpg}(内心:学生对物理失去兴趣真的是李院长的错吗?)


\end{scriptsize}
\end{minipage}
\ech
\end{frame}



\begin{frame}
\bch
下面我要开始向林树老师的风格转变……
\skipline

\ech
\end{frame}


\section{Basic Content of Cosmology}


\begin{frame}
\chtitle{宇宙学标准模型的概貌}
\bch
\begin{itemize}
\item{真空暴涨,宇宙加速膨胀(约持续$\sim 10^{-35}$s)}
\item{暴涨场衰变成其他场,热宇宙诞生,失去了真空能推斥力,宇宙开始减速膨胀并随着膨胀而冷却}
\item{标准模型粒子形成(这时宇宙年龄才约$1$s),这时因为温度极高,原子还都处于电离状态,自由电子到处和光子发生散射,使得宇宙非常地“不透明”。}
\item{宇宙冷却到大概$3000$K的时候,因宇宙中的光子能量不够电离氢原子,大部分自由电子都被氢原子核俘获。从此刻起光子即能自由穿行而不被自由电子散射。宇宙此时变得“透明”。此时宇宙的年龄约为几十万年。}
\item{宇宙冷却后的膨胀过程和宇宙的具体组成有关,经过观测我们推测宇宙中有暗物质(和普通物质一样起到吸引作用,使得宇宙的膨胀慢下来)和暗能量(类似真空能有排斥作用,使宇宙最近又开始加速膨胀)。现在宇宙的年龄是一百三十多亿年。}
\end{itemize}


\ech
\end{frame}

\begin{frame}
\chtitle{宇宙背景微波辐射(CMB)}
\bch
我们用电磁波观测能看到的最早的宇宙,是刚开始变成透明时的宇宙。那时的宇宙的光子自由穿行一百三十多亿年后被我们在地球上接收到。因为宇宙变透明后一直都在膨胀,这些光子波长被拉长后能量降低,到现在大约只有3K了。

\skipline
这个大约为3K的来自四面八方的光子,就是宇宙背景微波辐射(Cosmic Microwave Background)。
\ech
\end{frame}


\begin{frame}
\chtitle{CMB在各个方向上的温度起伏}
\bch
这个背景温度的数值对我们而言用处不大,有用的信息包含在各个方向的CMB的温度起伏中。这些起伏非常小,大概是$0.1$mK的量级。

\includegraphics[width=3in]{planck_masked.pdf}
\ech
\end{frame}

\begin{frame}
\chtitle{CMB的统计性质}
\bch
CMB温度起伏的根源可以一直追溯到最开始的真空暴涨。因为真空态的统计性质是完全已知的,我们就能计算CMB的各种统计性质。

\skipline
最简单的,例如我们可以固定一个角度$\theta$,然后计算夹角为$\theta$的任意两个方向的CMB温度起伏的乘积的统计平均(这称为两点关联函数)。因为标准模型是统计上各向同性的,所以这个结果只跟$\theta$有关,不妨记为$C(\theta)$。
\ech
\end{frame}

\begin{frame}
\chtitle{CMB的温度场两点关联函数测量}
\bch
习惯上我们喜欢讨论$C(\theta)$的勒让德变换$\mathcal{D}_\ell$。只要记住小的$\ell$对应大的$\theta$,这完全不影响我们理解下图:
  \includegraphics[width=3.5in]{planck2014_TT_Dl_NORES_bin30_w88mm.pdf}

  Planck卫星观测到的CMB温度场两点关联函数跟理论的比较。
\ech
\end{frame}

\begin{frame}
\chtitle{CMB的极化度测量}
\bch
实际上,对CMB我们不仅能观测它的温度,还能观测它的极化程度。(每个光子都有两个极化方向可以选择,相关知识你们以后在光学课会学到。)


这又提供了一种检验理论的办法。
\ech
\end{frame}


\begin{frame}
  \chtitle{Planck观测到的CMB温度场和极化场的两点关联函数}
  \includegraphics[width=3.5in]{planck2014_TE_Dl_NORES_bin30_w88mm.pdf}
\end{frame}

\begin{frame}
  \chtitle{Planck观测到的CMB极化场自身的两点关联函数}
  \includegraphics[width=3.5in]{planck2014_EE_Dl_NORES_bin30_w88mm.pdf}
\end{frame}


\begin{frame}
  \chtitle{不喜欢关联函数?没关系,我们可以把温度场图进行直接叠加。}
  \centering{
    \includegraphics[width = 2.in]{planck_masked.pdf}}
  
  {\hskip 0.2in} {\scriptsize unoriented stacking}     {\hskip 0.65in} {\scriptsize oriented stacking (Bond, Frolov, Huang)}
            
  \includegraphics[width = 2in]{ors_T_on_Tmax_unoriented.pdf}  
  \includegraphics[width = 2in]{ors_T_on_Tmax_oriented.pdf}
\end{frame}

\begin{frame}
  \frametitle{The CMB Polarization Anisotropy Map from Planck}
  \includegraphics[width=2in]{planckQhp_masked.pdf}
  \includegraphics[width=2in]{planckUhp_masked.pdf}  

  {\hskip 0.2in} {\scriptsize unoriented stacking}     {\hskip 0.65in} {\scriptsize oriented stacking (Bond, Frolov, Huang)}
  
  \includegraphics[width=2in]{ors_Qr_on_Tmax_unoriented.pdf}
  \includegraphics[width=2in]{ors_Q_on_Tmax_oriented.pdf}  
  
\end{frame}


\begin{frame}
\chtitle{暗物质和暗能量存在的直观证据}
\bch
\hskip 0.2in ACT数据 \hskip 0.6in   标准模型预言

\includegraphics[width=1.3in]{act_T_nu1_5a_lmin250_fpts.pdf}
\includegraphics[width=1.3in]{theory_lcdm_T_nu1_5a_lmin250_fpts.pdf}

如果无暗物质 \hskip 0.5in 如果无暗能量
\includegraphics[width=1.3in]{theory_nocdm_T_nu1_5a_lmin250_fpts.pdf}
\includegraphics[width=1.3in]{theory_noDE_T_nu1_5a_lmin250_fpts.pdf}

\ech
\end{frame}

\begin{frame}
\chtitle{其他观测?}
\bch
关于宇宙学的其他观测,如红移巡天,Type Ia超新星,Baryon Acoustic Oscillations (BAO),Ly$\alpha$ Forest等等。
我还是以后有机会再跟你们“漫谈”吧。
\ech
\end{frame}

\begin{frame}
\chtitle{宇宙学中还没有解决的问题}
\bch
\begin{itemize}
\item{暗物质粒子的本质是什么?}
\item{暗能量的本质是什么?}
\item{早期宇宙暴涨造成的原初引力波是否能测到?}
\item{能否重建早期宇宙暴涨场的具体粒子物理模型?}
\end{itemize}

这些都是宇宙学里比较热门的研究方向。


\ech
\end{frame}

\begin{frame}
\chtitle{实际的研究往往复杂艰难}

\bch
比如我最近在研究的Horndeski暗能量模型,它的作用量长成这样:
{\scriptsize
  \begin{eqnarray}
    S = && \int \sqrt{-g}d^4x \Big\{ G_2(\phi, X) + G_3(\phi, X)\Box\phi + G_4(\phi,X)R \nonumber \\
    && - 2G_{4,X}(\phi, X)\left[(\Box\phi)^2 - (\nabla^\mu\nabla^\nu\phi) (\nabla_\mu\nabla_\nu\phi)\right] \nonumber \\
    && + G_5(\phi, X)G_{\mu\nu}\nabla^\mu\nabla^\nu\phi +\frac{1}{3}G_{5X}(\phi, X)\times\left[(\Box\phi)^3 \right.\nonumber \\
    && \left.- 3 \Box\phi(\nabla^\mu\nabla^\nu\phi) (\nabla_\mu\nabla_\nu\phi) + 2(\nabla_\mu\nabla_\nu\phi)(\nabla^\sigma\nabla^\nu\phi) (\nabla_\sigma\nabla^\mu\phi)\right]\Big\}\nonumber
  \end{eqnarray}
}
\ech
\end{frame}

\begin{frame}
\chtitle{好了,不能再往下讲了……}
\bch
\skipline
\begin{minipage}{0.4\textwidth}

你再这样讲下去要招不到研究生了
\vskip 2in
\ 
\end{minipage}
\includegraphics[width=1.8in]{miaoshu.jpg}
\ech
\end{frame}


\begin{frame}
\bch
\centering{ \hskip 1.5in \Huge
谢谢}
\ech
\end{frame}

\end{document}



